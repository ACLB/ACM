\documentclass[serif]{beamer}
\usepackage[english]{babel}
\usepackage[noindent]{ctex}
\usepackage{graphicx}
\setbeamerfont{frametitle}{size=\small}
\usetheme{Madrid}
\title{FFT和NTT 简易讲解}
\author{绝尘}
\institute{山东理工大学}
\titlegraphic{\includegraphics[height=0.8in,width=4.5in]{logo.png}}
\date{\today}
\begin{document}

%%%% first page %%%%
\frame{\titlepage}

%%%% contents %%%%
\tableofcontents
\clearpage

%%%% section 1 %%%%
\section{基础知识}
\subsection{多项式乘法}
\begin{frame}
    \frametitle{多项式表达式}
    \begin{flushleft}
    \onslide<1->\centering{点值表达}
    \begin{description}
	\onslide<2->\item 1.什么是点值表达?
        \onslide<3->{\\ 比如一个多项式$$A(x) = \sum\limits_{i = 0}^{n-1}a_ix^i$$}
	\onslide<4->{我们在这里顺便假设一个列向量
	$$a = (a_0,a_1,a_2,\cdots,a_{n-1})$$}
	\onslide<5->{我们将$A(x)$看作是一个函数,然后在数轴上找到这些点
	$$x_1,x_2,x_3,\cdots,x_{n-1}$$}
	\onslide<6->{然后将这些点分别代入 $A(x)$,然后就可以得到$n$个点}  
        \onslide<7->{\item 2}
        \onslide<8>{\item 3}
    \end{description}
    \end{flushleft}
\end{frame}

%%%% section 2 %%%%
\section{}
\begin{frame}
    \frametitle{}

\end{frame}
\end{document}
%%%% end of file %%%%

